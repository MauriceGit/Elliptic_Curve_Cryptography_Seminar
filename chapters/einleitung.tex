\chapter{Einleitung}

Die folgende Ausarbeitung dient dem Ziel, einen allgemeinen Überblick zu schaffen über 
die Möglichkeit und Vorgehensweise des asymmetrischen Schlüsselaustauschs und Signatur mittels elliptischer Kurven.

Weiterhin wird die Fragestellung aufgegriffen, ob sich Kryptografie mittels elliptischer Kurven
aus Sicht von Effizienz- und Sicherheitsgründen besser eignet als andere asymmetrische, 
kryptografische Verfahren und wie sich bei gleicher Anzahl an Sicherheitsbits die Schlüssellänge der verschiedenen Algorithmen verhält.

Die Motivation für eine weitere Forschung neuer oder unterschiedlicher kryptografischer Methoden,
abseits der bekannten und verwendeten, ist besonders heutzutage wichtig, da es sehr schwer
abzusehen ist, wie lange bekannte Verfahren noch sicher verwendet werden können.

In der folgenden Arbeit werden keine neuen Erkenntnisse auf dem Gebiet der Kryptografie mittels elliptischer Kurven gewonnen.
Vielmehr geht es darum, das Konzept der Kryptografie mittels elliptischer Kurven vorzustellen und dieses anderen Verfahren gegenüber zu stellen.
