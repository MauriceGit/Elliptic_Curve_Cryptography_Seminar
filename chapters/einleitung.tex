\chapter{Einleitung}

Die folgende Ausarbeitung dient dem Ziel, einen allgemeinen Überblick zu schaffen über 
aktuelle asymmetrische Verschlüsselungsverfahren. Dabei wird ein Schwerpunkt auf der 
Verschlüsselung, Entschlüsselung und Signatur mittels elliptischer Kurven gelegt.

Weiterhin wird die Fragestellung aufgegriffen, ob sich Kryptografie mittels elliptischer Kurven
aus Sicht von Effizienz- und Sicherheitsgründen besser eignet als andere asymmetrische 
kryptografische Verfahren.

Die Motivation für eine weitere Forschung neuer oder unterschiedlicher kryptografischer Methoden
abseits der bekannten und verwendeten ist besonders heutzutage wichtig, da es sehr schwer
abzusehen ist, wie lange bekannte Verfahren noch sicher verwendet werden können.

In der folgenden Arbeit werden keine neuen Erkenntnisse auf dem Gebiet der Kryptografie mittels elliptischer Kurven gefunden.
Vielmehr geht es darum, eine Abgrenzung zu schaffen und verschiedene Verfahren einander gegenüber zu stellen.
Dabei wird auch auf praktische Beispiele eingegangen und anhand von Programmcode (Python mit \emph{Sage} \footnote{http://www.sagemath.org/de/}) erläutert.

