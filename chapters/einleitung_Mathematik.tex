\chapter{Zur Erinnerung}

In den folgenden Abschnitten werden einige grundlegende Algorithmen und Probleme erläutert, welche im Kontext der kommenden 
Kapitel genutzt werden und essentiell sind für die Funktionalität und Effektivität vieler kryptografischer Verfahren, wie RSA und Methoden mit elliptischen Kurven.

Es wird ein Grundverständnis der diskreten Mathematik (Definition von Ringen, Feldern, Gruppen) für das Verständnis der folgenden Kapitel vorausgesetzt.

\section{Prime Restklassengruppen}

Bei der Erzeugung asynchroner Verschlüsselungen, zu denen auch die Verfahren elliptischer Kurven gehören, befinden wir uns in mathematischen Gruppen.
Eine Gruppe ist meist eine endliche Menge an Elementen, kombiniert mit einem Operator. Also sei $(G,\cdotp)$ eine Gruppe.
Die Ordnung der Gruppe $G$ entspricht der Anzahl Elemente in $G$. Die Ordnung eines Elementes $\alpha \in G$ ist 
die niedrigste Zahl $n$ für die gilt: $\alpha^n = e$ mit $e =$ dem neutralen Element bezüglich des Operators $\cdotp$ in $G$.

Jede (zyklische) Gruppe besitzt einen Generator $\alpha$, für den gilt, dass $\forall a \in G \ \exists i \in  \mathbb{N}: \alpha^i = \alpha \cdotp \alpha \cdotp \cdotp \cdotp \alpha = a $. Sofern $G$ eine zyklische endliche Gruppe ist und $\alpha$ ein Generator dieser, so entspricht die Ordnung von $\alpha$ der von $G$.

Ein Beispiel einer primen Restklassengruppe ist $(\mathbb{Z}_p,\cdotp)$ mit der Voraussetzung, dass der Generator p eine Primzahl ist. Die Ordnung dieser Gruppe ist $p-1$.

Alle Operationen asynchroner Verschlüsselungen finden im Kontext meist primer Restklassengruppen statt. Hier 
muss keine weitere Überprüfung auf Eindeutigkeit und Zyklen erfolgen, da der Generator, eine Primzahl, die gewünschten
Eigenschaften der Gruppe bereits vorgibt.

\section{Diskretes Logarithmus Problem (DLP)}

Sei $ G $ eine zyklische Gruppe und $ g \in G $. Gegeben sei ein Element $h$ in der durch $g$ generierten Teilgruppe.
Das diskrete Logarithmus Problem für $G$ besteht nun darin, ein Element $m$ zu finden, welches die Gleichung $ h = g^m $ erfüllt.
Der kleinste Wert $m$, welches $h = g^m$ erfüllt, ist der sogenannte Logarithmus von $h$ im Bezug auf $g$. Also: $m = log_g(h)$.

Das diskrete Logarithmus Problem wird in der Kryptografie häufig als das zu Grunde liegende Problem für asynchronen Schlüsseltausch, digitale Signaturen oder Hash-Funktionen genutzt. Es eignet sich zur Nutzung in kryptografischen
Systemen durch den Umstand, dass sich im Sinne der Komplexitätstheorie der diskrete Logarithmus nur sehr ineffizient
berechnen lässt, während die Umkehrfunktion (auch im Sinne der Komplexitätstheorie) einfach berechnen lässt.
Diese Art von Funktion nennt sich auch Einwegfunktion und bildet die mathematische Grundlage aller asynchronen 
Verfahren der Kryptografie.	

\section{Anforderungen an asynchrone kryptografische Verfahren}

Asynchrone Verfahren der Kryptografie basieren auf dem Prinzip, dass zwei Kommunikationspartner verschlüsselt miteinander kommunizieren können ohne ein gemeinsames Geheimnis oder Schlüssel abgesprochen zu haben. Zwei entscheidende Anforderungen dabei sind, dass ein beliebiger Angreifer weder in der Lage sein soll die Kommunikation mitzulesen oder sie zu semantisch korrekt manipulieren.


